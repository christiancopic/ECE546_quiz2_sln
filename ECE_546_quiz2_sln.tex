\documentclass[12pt]{article}
\usepackage[margin=1in]{geometry}
\usepackage[all]{xy}
\usepackage{gensymb}


\usepackage{amsmath,amsthm,amssymb,color,latexsym}
\usepackage{geometry}        
\geometry{letterpaper}    
\usepackage{graphicx}

\newtheorem{problem}{Problem}

\newenvironment{solution}[1][\it{Solution}]{\textbf{#1. } }


\begin{document}
\noindent ECE/MSE 446/546\hfill Quiz 2 \\
SOLUTIONS \hfill Source Code: \texttt{\detokenize{https://github.com/christiancopic/ECE546_quiz2_sln}}

\hrulefill

\begin{problem}
Derive the pulling rate in the Czochralski crystal growth process.
\end{problem}
\begin{solution}
	Heat balance equation:
    \begin{align}
         L*\frac{dm}{dt} + k_L*\frac{T}{x_1} * A_1 = k_s*\frac{T}{x_2}*A_2
    \end{align}
    
    L = Latent heat of fusion\par
    $\frac{dm}{dt}$ = Rate of amount of Si freezing per unit time\par
    $k_L$ = thermal conductivity of melt\par
    $A_1$ = cross sectinal area of isotherm at $x_1$\par
    $\frac{dT}{dx_1}$ = temperature gradient at isotherm $x_1$\par
    $k_s$ = thermal conductivity of solid\par
    $A_2$ = cross sectional area of isotherm at $x_2$\par
    $\frac{dT}{dx_2}$ = temprature gradient at isotherm $x_2$\\

    The rate by which the crystal is pulled out of the melt is

    \begin{align}
        \frac{dm}{dt} = v_p * A * N
    \end{align}

    N = density of Si\par
    $v_p$ = Pull rate of crystal\\

   Put (2) into (1), with the assumption that $A_1 = A_2 = A$

   $$\implies L*v_p*A*N + k_L * \frac{dT}{dx_1}*A = k_s*\frac{dT}{dx_2}*A$$
   $$\implies L*v_p*A*N = k_s*\frac{dT}{dx_2}*A$$
   \begin{align}
        \implies v_{pMAX} = \frac{k_s}{LN} * \frac{dT}{dx_2}    
   \end{align}
   
    Heat loss due to radiation is given by Stefan-Boltzmann Law

    \begin{align}
        dQ = (2*\pi*r*dx)*(\sigma*\epsilon*T^4)
   \end{align}

   $2*\pi*r*dx$ = Radiating surface area\par
   $\sigma$ = Stefan-Boltzmann constant\par
   $\epsilon$ = Emissivity of Si\\

   The heat conducted up the crystal is given by

   \begin{align}
        Q = k_s*(\pi*r^2) * \frac{dT}{dx}
   \end{align}

   Differentiate (5) with respect to x

   $$\implies \frac{dQ}{dx} = k_s*(\pi*r^2)* \frac{d^2T}{dx^2}+\pi*r^2*\frac{dT}{dx}*\frac{dk_s}{dx} \cong k_s(\pi*r^2)*\frac{d^2T}{dx^2}$$

   \begin{align}
        \implies \frac{d^2T}{dx^2} - \frac{2*\sigma*\epsilon}{k_s*r}*T^4=0
   \end{align}

   Thermal conductivity of Si below 1000 $\degree$C is given by

   $$k_s = k_m* \frac{T_m}{T}$$

   where $k_m$ is the thermal conductivity of Si at melting point $T_m$

   \begin{align}
        \implies \frac{d^2T}{dx^2} - \frac{2*\sigma*\epsilon}{k_m*r*T_m}*T^5 = 0
   \end{align}

   Solve the differential equation (7) for T. Then differentiate T with respect with x and evaluate it at x=0 to get:

   $$V_{pMAX} = \frac{1}{L*N} * \sqrt{\frac{2*\sigma*\epsilon*k_m*T^5_m}{3*r}}$$

    
\end{solution} 

\begin{problem}
Derive the expression for the segregation coefficient in the Czochralski crystal growth process.
\end{problem}
\begin{solution}
	Equilibrium segregation coefficient

    $$k_o = \frac{C_S}{C_L}$$

    $C_S$ = Equilibrium concentration of the dopant in the solid\par
    $C_L$ = Equilibrium concentration of the dopant in the liquid near interface\\

    Doping concentration of liquid is

    $$C_L = \frac{S}{M_o - M}$$

    S = Amount of dopant remaining in the melt\par
    $M_o$ = Initial weight of grown crystal \par
    M = Crystal weight\\

    For incremental amount of weight dM of crystal, the corresponding reduction in dopant from the melt is 
    $$-dS = C_S*dM$$

    Combining our equations we yield
    $$\frac{dS}{S} = -k_o * \frac{dM}{M_o - M}$$

    Integrate and simplify to get

    $$C_S = C_o * k_o * (1-\frac{M}{M_o})^{k_o - 1}$$
    
\end{solution}



\begin{problem}
List the steps involved in the fabrication of a silicon wafer for logic device applications.
\end{problem}
\begin{solution}
	\begin{enumerate}
	    \item Wafer production
        \item Wafer slicing
        \item Surface polishing
        \item Oxidation
        \item Photolithography
        \item Etching
        \item Doping
        \item Thin film deposition
        \item Annealing
        \item Metallization
        \item Chemical Mechanical polishing
        \item Testing
        \item Dicing
        \item Packaging
	\end{enumerate}
\end{solution}
 
\end{document}
